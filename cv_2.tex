%%%%%%%%%%%%%%%%%%%%%%%%%%%%%%%%%%%%%%%%%
% Medium Length Graduate Curriculum Vitae
% LaTeX Template
% Version 1.1 (9/12/12)
%
% This template has been downloaded from:
% http://www.LaTeXTemplates.com
%
% Original author:
% Rensselaer Polytechnic Institute (http://www.rpi.edu/dept/arc/training/latex/resumes/)
%
% Important note:
% This template requires the res.cls file to be in the same directory as the
% .tex file. The res.cls file provides the resume style used for structuring the
% document.
%
%%%%%%%%%%%%%%%%%%%%%%%%%%%%%%%%%%%%%%%%%

%----------------------------------------------------------------------------------------
%	PACKAGES AND OTHER DOCUMENT CONFIGURATIONS
%----------------------------------------------------------------------------------------

\documentclass[margin, 7pt]{res} % Use the res.cls style, the font size can be changed to 11pt or 12pt here

\usepackage{charter} % Default font is the helvetica postscript font
%\usepackage{newcent} % To change the default font to the new century schoolbook postscript font uncomment this line and comment the one above

%Setup hyperref package, and colours for links
\usepackage{hyperref}
%\definecolor{linkcolour}{rgb}{0,0.2,0.6}
%\hypersetup{colorlinks,breaklinks,urlcolor=linkcolour, linkcolor=linkcolour}

\setlength{\textwidth}{5.1in} % Text width of the document

\begin{document}

%----------------------------------------------------------------------------------------
%	NAME AND ADDRESS SECTION
%----------------------------------------------------------------------------------------

\moveleft.5\hoffset\centerline{\LARGE\bf Amanda Pogue} % Your name at the top
 
\moveleft\hoffset\vbox{\hrule width\resumewidth height 1pt}\smallskip % Horizontal line after name; adjust line thickness by changing the '1pt'
 
\moveleft.5\hoffset\centerline{Department of Brain \& Cognitive Sciences, University of Rochester} % Your address
\moveleft.5\hoffset\centerline{Meliora Hall, Rochester, NY 14627}
\moveleft.5\hoffset\centerline{\href{http://amandapogue.github.io}{http://amandapogue.github.io}}
\moveleft.5\hoffset\centerline{\href{mailto:apogue@ur.rochester.edu}{apogue@ur.rochester.edu}}

%----------------------------------------------------------------------------------------

\begin{resume} 

%----------------------------------------------------------------------------------------
%	EDUCATION SECTION
%----------------------------------------------------------------------------------------

\section{EDUCATION}

{\bf PhD, Brain \& Cognitive Sciences}, expected 2018 \\
University of Rochester, Rochester, NY \\
Advisors: Michael K. Tanenhaus, Chigusa Kurumada \medskip \\
{\bf MA, Brain \& Cognitive Sciences}, 2016 \\
University of Rochester, Rochester, NY \\
Committee: Michael K. Tanenhaus, Chigusa Kurumada, T. Florian Jaeger \medskip \\
{\bf MA, Developmental Psychology}, 2013 \\
University of Waterloo, Waterloo, ON \\
Committee: Mathieu Le Corre, Ori Friedman, Stephanie Denison  \\
Thesis: \href{https://uwspace.uwaterloo.ca/handle/10012/7822}{What makes a few more than a lot: a study of context-dependent quantifiers} \medskip \\
{\bf Hon. BSc, Psychology, Cognitive Science, Philosophy}, 2009 \\
University of Toronto, Toronto, ON \\

 
%----------------------------------------------------------------------------------------
%	Publications
%----------------------------------------------------------------------------------------
 
 \section{PUBLICATIONS}

{\bf Pogue, A.},\& Tanenhaus, M.K. (in prep). Leaning and using (un)certainty.\medskip \\
Orth, W., {\bf Pogue, A.}, \& Kurumada, C. (in prep). The role of context in the interpretation of scalar adjectives. \medskip \\
{\bf Pogue, A.}, Tanenhaus, M.K., \& Kurumada, C. (in prep). The role of uncertainty and feedback in learning to generalize to talker-specific utterances based on informativity.\medskip \\
{\bf Pogue, A.}, Brown-Schmidt, S., \& Kurumada, C. (in prep). The role of causal attribution in the generalization of pragmatic informativity assumptions. \medskip \\
{\bf Pogue, A.}, Kurumada, C., \& Tanenhaus, M.K. (2016). Talker-specific generalization of pragmatic inferences based on under- and over-informative prenominal adjective use. {\sl Frontiers in Psychology,  6:2035}. \medskip \\
Hartshorne, J.K., {\bf Pogue, A.}, \& Snedeker, J. (2015). Love is hard to understand: The relationship between causativity and transitivity in the acquisition of emotion verbs. {\sl Journal of Child Language, 42}, 467-504. \medskip \\
Vouloumanos, A., Onishi, K., \& {\bf Pogue, A.} (2012). Twelve-month-old infants recognize that speech can communicate unobservable intentions. {\sl Proceedings of the National Academy of Sciences, 109}, 12933-12937. \medskip \\
Brooks, N., {\bf Pogue, A.}, \& Barner, D. (2011). Piecing together numerical language: Children's use of default units in early counting and quantification. {\sl Developmental Science, 14}, 44-57. \\

%----------------------------------------------------------------------------------------
%	Invited Talks
%----------------------------------------------------------------------------------------
 
 \section{INVITED TALKS}
Learning about speakers and the world: a delicate balance of expectations and experience. Presented at The Language Lab at MIT, October 2017. \medskip \\
How Adults and Children learn about speakers and the world: a delicate balance of expectations and experience. Presented at The Language Learning Lab at Boston College, October 2017.\medskip \\
\pagebreak
Referential Expressions: generating predictions online, perhaps a useful step for the future of robotics. Presented at the Humans to Robots Lab at Brown University, May 2017. \medskip \\


%----------------------------------------------------------------------------------------
%	Conferences
%----------------------------------------------------------------------------------------
 
 \section{CONFERENCE PROCEEDINGS}
 
{\bf Pogue, A.}, Kurumada, C., \& Tanenhaus, M.K. (2015). Speaker-specific generalization of pragmatic inferences based on prenominal adjectives. In Noelle, D. C., Dale, R., Warlaumont, A. S., Yoshimi, J., Matlock, T., Jennings, C. D., \& Maglio, P. P. (Eds.), {\sl Proceedings of the 37th Annual Conference of the Cognitive Science Society (pp. 1889-1894)}. Austin, TX: Cognitive Science Society.\medskip \\
{\bf Pogue, A.}, Jalabi, A., \& Le Corre, M. (2012). Why is a few sometimes a lot? In N. Miyake, D. Peebles, \& R. P. Cooper (Eds.), {\sl Proceedings of the 34th Annual Conference of the Cognitive Science Society (pp. 2192-2197)}. Austin, TX: Cognitive Science Society. \medskip \\
Brooks, N., {\bf Pogue, A.}, \& Barner, D. (2010). Piecing together numerical language: Children's use of default units in early counting and quantification. In K. Franich, K. M. Iserman, \& L. L. Keil (Eds.) {\sl Proceedings of the 34th Annual Boston University Conference on Language Development}. Somerville, MA: Cascadilla Press. \medskip \\
{\bf Pogue, A.}, Hartshorne, J.K., \& Snedeker, J., (2009). Developmental Evidence for a Canonical Syntax-Semantics Mapping for Verbs of Psychological State [Abstract]. In N.A. Taatgen \& H. van Rijn (Eds.), {\sl Proceedings of the 31st Annual Conference of the Cognitive Science Society (p. 972)}. Austin, TX: Cognitive Science Society. \medskip \\
Melgoza, V., {\bf Pogue, A.}, \& Barner, D. (2008). A Broken Fork in the Hand is Worth Two in the Grammar: Evidence for a Discrete Physical Object Bias in Quantifiers and Plural Nouns. In B. C. Love, K. McRae, \& V. M. Sloutsky (Eds.), {\sl Proceedings of the 30th Annual Conference of the Cognitive Science Society (pp. 1580-1585)}. Austin, TX: Cognitive Science Society. \\

%----------------------------------------------------------------------------------------
%	Posters and Presentations 
%----------------------------------------------------------------------------------------
 
 \section{POSTERS \& PRESENTATIONS}
Gardner, B., Clark, M., {\bf Pogue, A.}, \& Kurumada, C. (2018). Real-time pragmatic processing with a novel lexicon. Poster to be presented at the CUNY Conference on Sentence Processing. Davis, CA. \medskip \\
{\bf Pogue, A.}, \& Tanenhaus, M.K. (2018). Communicating, interpreting, and learning from uncertainty: Could be a promising approach. Poster to be presented at the CUNY Conference on Sentence Processing. Davis, CA. \medskip \\
{\bf Pogue, A.}, \& Tanenhaus, M.K. (2017). Exploring how speakers mark, and listeners assess, certainty. Poster presented at Experimental Pragmatics (XPrag 2017). Cologne, Germany. \medskip \\
{\bf Pogue, A.}, Brown-Schmidt, S., \& Kurumada, C. (2017). Causal attributions in the adaptation of pragmatic informativity assumptions. Poster presented at the CUNY Conference on Sentence Processing. Boston, MA. \medskip \\
Orth, W., {\bf Pogue, A.}, \& Kurumada, C. (2017). The role of context and information in the interpretation of scalar adjectives.  Poster presented at the CUNY Conference on Sentence Processing. Boston, MA. \medskip \\
Orth, W., {\bf Pogue, A.}, \& Kurumada, C. (2017). Contextual Factors In Child Adjective Interpretation. Poster presented at the National Conference on Undergraduate Research. Nashville, TN. \medskip \\
{\bf Pogue, A.}, Kurumada, C., \& Tanenhaus, M.K. (2017). Preschoolers' pragmatic generalization based on scalar adjective use. Poster to be presented at the CSLI workshop: Bridging Computational and Psycholinguistic Approaches to the Study of Meaning. Stanford, CA. \medskip \\
{\bf Pogue, A.}, Brown-Schmidt, S., \& Kurumada, C. (2017). The role of causal attribution in the generalization of pragmatic informativity assumptions. Poster presented at the CSLI workshop: Bridging Computational and Psycholinguistic Approaches to the Study of Meaning. Stanford, CA. \medskip \\
Kurumada, C., \& {\bf Pogue, A.} (2016). \href{http://amandapogue.github.io/Qualtrics_Tutorial/}{Running psycholinguistic experiments online: Using survey websites to conduct comprehension studies}. Tutorial at The Workshop on Experimental Approaches to East Asian Linguistics. Manoa, HI. \medskip \\
{\bf Pogue, A.}, Kurumada, C., \& Tanenhaus, M.K. (2015). Speaker-based generalization of quantity implicature in preschoolers. Presentation given at the 40th Annual Boston University Conference on Language Development. Boston, MA. \medskip \\
{\bf Pogue, A.}, Kurumada, C., \& Tanenhaus, M.K. (2015). Speaker-specific generalization of pragmatic inferences based on prenominal adjectives. Presentation at the 37th Annual Conference of the Cognitive Science Society. Pasadena, CA. \medskip \\
{\bf Pogue, A.}, Kurumada, C., \& Tanenhaus, M.K. (2015). Speaker-specific pragmatic generalizations based on under- vs. over-informative utterances. Presentation at Experimental Pragmatics 2015. Chicago, IL. \medskip \\
{\bf Pogue, A.}, Kurumada, C., \& Tanenhaus, M.K. (2015). Exploring expectations based on speaker-specific variation in informativity. Presentation at Formal and experimental pragmatics: methodological issues of a nascent liaison. Berlin, Germany. \medskip \\
{\bf Pogue, A.}, Jalabi, A., \& Le Corre, M. (2012). Why is a few sometimes a lot? Poster at the 34th Annual Conference of the Cognitive Science Society. Sapporo, Japan. \medskip \\
{\bf Pogue, A.}, Jalabi, A., \& Le Corre, M. (2012). Learning the order of things: How children develop an adult-like quantifier scale. Poster at the 42nd Annual Meeting of the Jean Piaget Society. Toronto, ON. \medskip \\
{\bf Pogue, A.}, Jalabi, A., \& Le Corre, M. (2011). When is ``a lot" more than ``a few"? Poster at the Seventh Biennial Meeting of the Cognitive Development Society. Philadelphia, PA. \medskip \\
Brooks, N., {\bf Pogue, A.}, \& Barner, D. (2009). Piecing together numerical language: Children's use of default units in early counting and quantification. Presentation given at the 34th Annual Boston University Conference on Language Development. Boston, MA. \medskip \\
{\bf Pogue, A.}, Hartshorne, J.K., \& Snedeker, J., (2009). Developmental Evidence for a Canonical Syntax-Semantics Mapping for Verbs of Psychological State. Poster at the 31st Annual Conference of the Cognitive Science Society. Amsterdam, Netherlands. \medskip \\
Hartshorne, J.K., {\bf Pogue, A.}, \& Snedeker, J. (2009). Fear and Loathing in Discourse Coherence: Children use the argument structure of psychological predicates in interclausal pronoun interpretation. Poster presented at the 3rd Annual Experimental Pragmatics Conference. Lyons, France. \medskip \\
Hartshorne, J.K., {\bf Pogue, A.}, \& Snedeker, J (2009). Who is she and why is she so scary? Psych verbs and the mapping from semantics to syntax. Poster at the 2009 Society for Research in Child Development Biennial Meeting, Denver, CO. \medskip \\
Melgoza, V., {\bf Pogue, A.}, \& Barner, D. (2008). A Broken Fork in the Hand is Worth Two in the Grammar: Evidence for a Discrete Physical Object Bias in Quantifiers and Plural Nouns. Poster at the 30th Annual Conference of the Cognitive Science Society. Washington, DC. \medskip \\
Melgoza, V., \& {\bf Pogue, A.} (2008). A Spatio-Temporal Bias in Children's Interpretation of Quantifiers and Plural Nouns. Presentation given at the Toronto Undergraduate Linguistics Conference 2008, Toronto, ON. \\

%------------------------------------------------------------------------
%	Awards
%------------------------------------------------------------------------
 
\section{AWARDS}

\href{https://www.nsf.gov/awardsearch/showAward?AWD_ID=1727336}{NSF Doctoral Dissertation Improvement Award}: The interaction of  \hfill 2017-2019 \\
expectations and evidence in pragmatics \\
CUNY 2017 Conference Travel Award \hfill 2017 \\
URochester Graduate Student Association Conference Travel Funding \hfill 2015 \\
Robert J. Glushko and Pamela Samuelson Foundation Student Travel  \hfill 2015 \\
Grant \\
National Science and Engineering Research Council Post-Graduate \hfill 2013 - 2015 \\ 
Scholarship D2  \\
Ontario Graduate Scholarship \hfill 2012 - 2013 \\
University of Waterloo President's Graduate Scholarship \hfill 2012 - 2013 \\
University of Waterloo Travel Scholarship \hfill 2011 - 2013 \\
University of Waterloo Arts Graduate Scholarship \hfill 2010 - 2012 \\
Linguistic Society of America Fellowship \hfill 2011 \\
University College C.L. Burton Open Scholarship \hfill 2009 \\
University College Merit Award \hfill 2009 \\
New College Student Council Grant \hfill 2008 \\
James McAdams Perron Scholarship \hfill 2008 \\

%----------------------------------------------------------------------------------------
%	REVIEWING
%----------------------------------------------------------------------------------------

\section{REVIEWING}
Cognitive Science Society  \\
Cognition \\


%------------------------------------------------------------------------
%	Teaching
%------------------------------------------------------------------------

\section{TEACHING} 
{\bf Teaching Assistant}, Foundations of Cognitive Science \hfill Spring 2017 \\
{\bf Guest Lecturer}, Language Development \hfill Spring 2016 \\
{\bf Guest Lecturer}, Intro to Pragmatics \hfill Spring 2016 \\
{\bf Teaching Assistant}, Language Development	\hfill Spring 2016 \\
{\bf Guest Lecturer}, Language Development \hfill Spring 2015 \\
{\bf Guest Lecturer}, Intro to Pragmatics \hfill Spring 2015 \\
{\bf Teaching Assistant}, Language Development	\hfill Spring 2015 \\
{\bf Teaching Assistant}, Introduction to Developmental Psychology \hfill	 Fall 2012 \\
{\bf Teaching Assistant}, Cognitive Development 	\hfill Winter 2012 \\
{\bf Teaching Assistant}, Research in Developmental Psychology	\hfill Fall 2011 \\
{\bf Teaching Assistant}, Introduction to Developmental Psychology 	\hfill Winter 2011 \\
{\bf Teaching Assistant}, Language Development 	\hfill Fall 2010 \\


%------------------------------------------------------------------------
%	Student Supervisation
%------------------------------------------------------------------------

\section{SUPERVISATION}

{\bf Bethany Gardner}, Independent Study Student	\hfill 2017 - present \\
{\bf Cameron Morgan}, Independent Study Student	\hfill 2017 - present \\
{\bf Ashley Andino}, Research Assistant	\hfill 2016 - present \\
{\bf Sahed Martinez}, Research Assistant	\hfill 2016 - present \\
{\bf Crystal Lee}, Research Assistant	\hfill 2016 - present \\
{\bf Wesley Orth}, Independent Study Student	\hfill 2016 - 2017 \\
{\bf Stephen Powell}, Independent Study Student	\hfill 2014 - 2017 \\
{\bf Valerie Langois}, Independent Study Student	\hfill 2015 - 2016 \\
{\bf Rocco Porcellio}, Research Assistant	\hfill 2014 - 2016 \\
{\bf Teigan Ruster}, Research Assistant	\hfill 2014 - 2015 \\
{\bf Rachel Oliver}, Research Assistant	\hfill 2011 - 2013 \\
{\bf Adel Jalabi}, Research Assistant	\hfill 2010 - 2013 \\

%------------------------------------------------------------------------
%	PROFESSIONAL ASSOCIATIONS
%------------------------------------------------------------------------

\section{PROFESSIONAL ASSOCIATIONS}

American Psychological Association \hfill 2016 - present \\
University of Rochester Center for Language Sciences	\hfill 2013 - present \\
Cognitive Science Society	\hfill 2008 - present \\ 
Society for Research in Child Development   	\hfill 2009 - 2013 \\
Cognitive Development Society	\hfill 2011 - 2013 \\
Linguistic Society of America   \hfill	 2011-2013 \\

%----------------------------------------------------------------------------------------
%	VOLUNTEERING
%----------------------------------------------------------------------------------------

\section{VOLUNTEERING} 

Organizer, CLS Interdisciplinary Talk Series \hfill 2017 - present \\
Organizer, Kinder / HLP Lab RA Meetings \hfill 2016 - present \\
Volunteer, Refugee Resettlement, Catholic Family Center \hfill 2017 - present \\
GSA Travel Award Reviewer, UR Graduate Student Association \hfill 2016 - present \\
Organizer, Experimental Semantics and Pragmatics Reading Group \hfill 2014 - present \\
Social Attache, Center for Language Sciences	\hfill 2014 - present \\
Student Science Educator, Rochester Museum of Science \hfill 2016 - 2016 \\
Volunteer Educator, Sweet Beez - Bee Education and Conservation	\hfill 2015 - 2016 \\
Volunteer, Queen Street Commons, The Working Centre	\hfill2012 - 2013 \\
Volunteer, Grow Garden, The Working Centre	\hfill 2012 - 2013 \\
Alumni Mentor, University College at the University of Toronto 	\hfill 2011 - 2013 \\
President, Graduate Association of Students in Psychology (GASP) 	\hfill 2012 - 2013 \\
Secretary, Graduate Association of Students in Psychology (GASP)	\hfill 2011 - 2012 \\
Co-founder, UW Graduate Cognitive Science Society \hfill	 2011 - 2012 \\
Student Organizer, Graduate Student Research Conference	\hfill 2010 - 2011 \\
UCOC Mentor Deputy, University College Literary and Athletic Society	\hfill 2007 - 2008 \\
Services Deputy, University College Literary and Athletic Society	\hfill 2006 - 2007

%----------------------------------------------------------------------------------------

\end{resume}
\end{document}